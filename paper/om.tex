\documentclass{article}
\usepackage{amsmath}
\bibliographystyle{alpha}

\title{Orthotope Machine}
\author{Takayuki Muranushi}
\begin{document}
\maketitle
\begin{quote}
  In geometry, an orthotope (also called a hyperrectangle or a box) is
  the generalization of a rectangle for higher dimensions, formally
  defined as the Cartesian product of intervals.
\end{quote}

\section{Introduction}

This document describes the {\em Orthotope Machine}, a virtual machine
that operates on multidimensional arrays. The Orthotope Machine is one
of the main components for Paraiso project. The goal of Paraiso
project is to create a high-level programming language for massively
parallel, partial differential equation solving.

Data Parallel Haskell\cite{nested-data-parallelism}.

Stencil.

High Performance Fortran, which James Stone described me in Taiwan as
a "Failed Project." 

The unique point of Paraiso compared to those projects is its focus on
multidimensional array computations. Multidimensional
arrays are different from nested arrays, 

The Orthotope Machine is designed to model and utilize these
characteristics of the computational domain.

\bibliography{paraiso}
\end{document}

