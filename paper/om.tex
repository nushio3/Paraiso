\documentclass{article}
\usepackage{amsmath}

%% \setlength{\oddsidemargin}{-0.4mm} 
%% \setlength{\evensidemargin}{\oddsidemargin}
%% \setlength{\textwidth}{170mm} 
\setlength{\textheight}{44\baselineskip}
\addtolength{\textheight}{\topskip}
\setlength{\voffset}{-0.6in}


\bibliographystyle{alpha}

\title{Orthotope Machine}
\author{Takayuki Muranushi}
\begin{document}
\maketitle
\begin{quote}
  In geometry, an orthotope (also called a hyperrectangle or a box) is
  the generalization of a rectangle for higher dimensions, formally
  defined as the Cartesian product of intervals.
\end{quote}

\section{Introduction}

This document describes the {\em Orthotope Machine}, a virtual machine that
operates on multidimensional arrays. The Orthotope Machine is one of the main
components for Paraiso project. The goal of Paraiso project is to create a
high-level programming language for generating massively parallel, explicit
solver algorithms of partial differential equations.

From computational viewpoint, explicit solvers of partial differential
equations belongs to the algorithm category called stencil codes. Stencil
codes are algorithms that updates the array, each element accessing the nearby
elements in the same pattern. Stencil codes are commonly used algorithms in
fields such as solving partial differential equations and image
processing. Code generations and automated tuning for stencil codes has been
studied \cite[e.g.]{Datta:EECS-2009-177, Datta:2008:SCO:1413370.1413375}.

There are many methods other than stencil codes for solving partial
differential equations. They have different merits. A notable project in
progress is Liszt \cite{Chafi:2010:LVH:1932682.1869527}, an embedded DSL in
programmin language Scala, designed for generating hydrodynamics solver on
unstructured mesh.

Data Parallel Haskell \cite{nested-data-parallelism}.



High Performance Fortran was a very promising approach to introduce a
high-level parallelism in Fortran but, as James Stone described me in Taiwan,
it was a failed project \cite{Kennedy:2007:RFH:1238844.1238851}.
DEQSOL\cite{SAGAWANOBUTOSHI:1989-01-15,Kon'no:1986:AIS:324493.325029} was
another project which had design similar to that of Paraiso. The language was
initially designed for Hitachi vector machines. The extension of DEQSOL for
parallel vector machines has been planned \cite{SagawaNobutoshi:1989-03-15}
but seemingly did not realize.

The unique point of Paraiso compared to those projects is its focus on
computational domains that utilize localized access to multidimensional
arrays.

Multidimensional arrays are different from nested arrays.

The Orthotope Machine is designed to capture and utilize these
characteristics of the localized multidimensional array computations.

\section{Explicit Solvers of Partial Differential Equations}

\section{Overall Design of Paraiso, and Orthotope Machine's Role in it}


\section{Definitions of Orthotope, Orthotree, and Distributed Orthotope}

\section{API for Orthotope Machine}


\section{Hardware Model}
\section{Instruction Set}
\subsection{Instruction Set for Primodial Orthotope Machine}
\subsection{Instruction Set for Distributed Orthotope Machine}

\section{Possible Program Transformations}
\subsection{Common Techniques}
\subsection{Timestep Fusion}
\subsection{Manual Cache}
\subsection{Synchronization Insertion}
\subsection{Trapezium Splitting}
\subsection{Parallelogram Splitting}



\bibliography{paraiso}
\end{document}

